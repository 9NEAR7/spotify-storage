\documentclass{article}
\usepackage[utf8]{inputenc}

\title{Proyecto Aplicación con Firebase y React}
\author{Cristian Terán}
\date{June 2022}

\begin{document}

\maketitle

\section{Introducción}
Este proyecto está relaizado con el objetivo aprender a usar el servicio de Firebase con su base de datos, almacenamiento(Storage) para la materia de base de datos no convencional.
Para este proyecto se utilizaron varias herramientas para el backend a tráves de firebase, para el front se utilizó la tecnologia de REACT.

\section{Firebase}
Firebase básicamente es una plataforma móvil diseñada y creada por Google, teniendo como principal función desarrollar y facilitar la creación de aplicaciones para dispositivos móviles que cuenten con una alta calidad a pesar de su rápida elaboración; esto con la finalidad de que se pueda incrementar la base de datos de usuarios y de esta manera incrementar la monetización de dicha app (ganar más dinero).

Esta plataforma se encuentra alojada en la nube y, por ende, está disponible para diferentes plataformas como Android, iOS, y web. Así mismo, cuenta con diversas funciones para que cualquier desarrollador pueda combinar y adaptar la plataforma a medida de sus necesidades.

\subsection{Para qué sirve Firebase}
la función principal de Firebase es hacer que el ciclo de desarrollo, tanto de aplicaciones móviles como de web, se lleve a cabo de manera armónica y sencilla, lo que conlleva a que, tanto el trabajo como el tiempo empleado sean rápidos, sin dejar de un lado la calidad que debe caracterizar a todo proyecto.

Es importante mencionar que Firebase es ideal para que los desarrolladores no requieran dedicar tanto tiempo a la construcción del backend (tanto en desarrollo, como en mantenimiento).

Así mismo, gracias a que su gran variedad de herramienta y su simple uso, teniendo en consideración que su agrupación simplifica las tareas de gestión bajo una misma plataforma podemos dividir en cuatro (04) grupos las finalidades principales de Firebase destacando las siguientes:

Análisis, Desarrollo, Crecimiento y Monetización

\subsection{Características de Firebase}
A continuación detallaremos brevemente algunas de las características de Firebase:
\dots  \dots
\begin{itemize}
\item Es multiplataforma: Soportada por Android, iOS y web.
\item Monetización: A través de Firebase podemos ganar dinero esto a través de AdMob con anuncios y publicidad.
\item Gran poder de crecimiento: Gracias a la fácil gestión de los usuarios de las aplicaciones es posible obtener un alto crecimiento según los objetivos planteados. Esta herramienta cuenta con el valor añadido de que podemos llegar a nuevos usuarios con el envío de notificaciones e invitaciones.
\item  Es Ágil: Ofrece el desarrollo y gestión de apps multiplataforma gracias a sus APIs integradas a SDK tanto para JavaScript como para iOS y Android, permitiendo gestionar diferentes aplicaciones sin la necesidad de la salir de la plataforma.
\end{itemize}

\subsection{Ventajas de Firebase}
De forma global podemos tomar como una gran ventaja a la Cloud Storage, la cual nos permite contar con una base de datos para que el usuario pueda contar con un espacio de almacenamiento y compartir imágenes y ficheros; sin recurrir a bases de datos propias. Así mismo, utiliza Cloud Functions, permitiéndonos ahorrar infraestructura de backend.

En líneas generales, sus funcionalidades se complementan entre sí a pesar de ser variadas. Firebase facilita los eventos en cuanto a el envío de notificaciones que gracias a su simpleza son de fácil uso y nos permiten centrar la atención de los usuarios. Cuenta con SSL (Secure Sockets Layer).

Por otra parte, cuenta con un panel central muy intuitivo, simple y de fácil acceso. En este panel podemos contar con múltiples opciones, de las cuales destacaremos el punto de englobar la analítica que nos permitirá tomar decisiones más acertadas en cada una de las fases del proyecto.

Otra de sus grandes ventajas es que permite a los desarrolladores centrar su atención y esfuerzo en aspectos en específico (como por ejemplo: llevar a cabo el desarrollo del frontend y dejar al backend en segundo plano), gracias a todas la herramientas que fomentaran el crecimiento y serán parte del éxito de nuestro proyecto.

En este mismo orden de ideas podemos destacar que cuenta con una amplia comunidad y documentación de calidad en la web.

\subsection{Cómo usar Firebase}

Inicialmente debes tener en cuenta a qué app la deseas agregar, bien sea web, Android o iOS, ya que los primeros pasos serán la instalación y configuración de Firebase al proyecto, para posteriormente llevar a cabo todo el proceso de desarrollo “acompañado” y crecimiento del proyecto (app lanzamiento, correcciones, mejoras y actualizaciones).

Una vez instalado y configurado podemos crear un flujo integrado de trabajo, nos ofrece la opción de personalizar la pantalla de bienvenida para los usuarios.

Para aprender a utilizar Firebase a cabalidad puedes validar toda la documentación en su sitio web oficial a través del siguiente enlace, que de forma muy explícita, te ayudará paso a paso a que te inicies en este mundo.

No obstante, si deseas aplicarlo a Angular en OpenWebinars contamos con el Curso de Firebase y Angular.

\subsection{Relación entre los proyectos de Google Could y Firebase}
Cuando creamos un proyecto en Firebase console estamos trabajando con Google Cloud; los proyectos en la nube pueden ser consideradios como contenerdore de datos, códigos, servicios, etc. En vista de que un proyecto de Firebase es un proyecto de Google Cloud, ocurre lo siguiente:


\begin{itemize}
\item Los proyectos que se observan en Firebase console también aparecen en Google Cloud Console y en la Consola de API de Google.
\item En cuanto a los permisos y la facturación de los proyectos se comparten entre Firebase y Google Cloud.
\item Los identificadores únicos de un proyecto (como el número y el ID) se comparten entre Firebase y Google Cloud.
\item Puedes usar productos y API tanto de Firebase como de Google Cloud en un proyecto.
\end{itemize}

 \subsection{Storage en Firebase}
 
 Cloud Storage para Firebase es un servicio de almacenamiento de objetos potente, simple y rentable construido para el escalamiento de Google. Los SDK de Firebase para Cloud Storage agregan la seguridad de Google a las operaciones de carga y descarga de archivos de tus apps de Firebase, sin importar la calidad de la red.
Puedes usar nuestros SDK para almacenar imágenes, audio, video y otros tipos de contenido generado por el usuario. En el servidor, puedes usar las API de Google Cloud Storage para acceder a los mismos archivos.


\subsection{Fron con React}
Es una librería open source de JavaScript para desarrollar interfaces de usuario. Fue lanzada en el año 2013 y desarrollada por Facebook, quienes también la mantienen actualmente junto a una comunidad de desarrolladores independientes y compañías.

Hoy en día muchas empresas de primer nivel utilizan React para el desarrollo de sus aplicaciones, y es que entre ellas podemos encontrar Facebook, Instagram y el cliente web de WhastApp (todas propiedad de Facebook), y otras como AirBnb, Uber, Netflix, Twitter, Reddit o Paypal.

Desde su lanzamiento, su uso ha ido incrementando notablemente, convirtiéndose, a día de hoy, en una de las tecnologías front-end más utilizadas.


\end{document}
